\documentclass[$if(fontsize)$$fontsize$,$endif$$if(lang)$$babel-lang$,$endif$$if(papersize)$$papersize$,$endif$$for(classoption)$$classoption$$sep$,$endfor$]{$documentclass$}
$if(fontfamily)$
\usepackage[$fontfamilyoptions$]{$fontfamily$}
$else$
\usepackage{lmodern}
$endif$
$if(linestretch)$
\usepackage{setspace}
\setstretch{$linestretch$}
$endif$
\usepackage{amssymb,amsmath}
\usepackage{ifxetex,ifluatex}
\usepackage{fixltx2e} % provides \textsubscript
\ifnum 0\ifxetex 1\fi\ifluatex 1\fi=0 % if pdftex
  \usepackage[$if(fontenc)$$fontenc$$else$T1$endif$]{fontenc}
  \usepackage[utf8]{inputenc}
$if(euro)$
  \usepackage{eurosym}
$endif$
\else % if luatex or xelatex
  \ifxetex
    \usepackage{mathspec}
  \else
    \usepackage{fontspec}
  \fi
  \defaultfontfeatures{Mapping=tex-text,Scale=MatchLowercase}
  \newcommand{\euro}{€}
$if(mainfont)$
    \setmainfont[$mainfontoptions$]{$mainfont$}
$endif$
$if(sansfont)$
    \setsansfont[$sansfontoptions$]{$sansfont$}
$endif$
$if(monofont)$
    \setmonofont[Mapping=tex-ansi$if(monofontoptions)$,$monofontoptions$$endif$]{$monofont$}
$endif$
$if(mathfont)$
    \setmathfont(Digits,Latin,Greek)[$mathfontoptions$]{$mathfont$}
$endif$
$if(CJKmainfont)$
    \usepackage{xeCJK}
    \setCJKmainfont[$CJKoptions$]{$CJKmainfont$}
$endif$
\fi
% use upquote if available, for straight quotes in verbatim environments
\IfFileExists{upquote.sty}{\usepackage{upquote}}{}
% use microtype if available
\IfFileExists{microtype.sty}{%
\usepackage{microtype}
\UseMicrotypeSet[protrusion]{basicmath} % disable protrusion for tt fonts
}{}
$if(geometry)$
\usepackage[$for(geometry)$$geometry$$sep$,$endfor$]{geometry}
$endif$
\makeatletter
\@ifpackageloaded{hyperref}{}{%
\ifxetex
  \usepackage[setpagesize=false, % page size defined by xetex
              unicode=false, % unicode breaks when used with xetex
              xetex]{hyperref}
\else
  \usepackage[unicode=true]{hyperref}
\fi
}
\@ifpackageloaded{color}{
    \PassOptionsToPackage{usenames,dvipsnames}{color}
}{%
    \usepackage[usenames,dvipsnames]{color}
}
\makeatother
\hypersetup{breaklinks=true,
            bookmarks=true,
            pdfauthor={$author-meta$},
            pdftitle={$title-meta$},
            colorlinks=true,
            citecolor=$if(citecolor)$$citecolor$$else$blue$endif$,
            urlcolor=$if(urlcolor)$$urlcolor$$else$blue$endif$,
            linkcolor=$if(linkcolor)$$linkcolor$$else$magenta$endif$,
            pdfborder={0 0 0}
            $if(hidelinks)$,hidelinks,$endif$}
\urlstyle{same}  % don't use monospace font for urls
$if(lang)$
\ifxetex
  \usepackage{polyglossia}
  \setmainlanguage[$polyglossia-lang.options$]{$polyglossia-lang.name$}
$for(polyglossia-otherlangs)$
  \setotherlanguage[$polyglossia-otherlangs.options$]{$polyglossia-otherlangs.name$}
$endfor$
\else
  \usepackage[shorthands=off,$babel-lang$]{babel}
\fi
$endif$
$if(natbib)$
\usepackage{natbib}
\bibliographystyle{$if(biblio-style)$$biblio-style$$else$plainnat$endif$}
$endif$
$if(biblatex)$
\usepackage{biblatex}
$for(bibliography)$
\addbibresource{$bibliography$}
$endfor$
$endif$
$if(listings)$
\usepackage{listings}
\usepackage{color}
\definecolor{OliveGreen}{cmyk}{0.64,0,0.95,0.40}
\definecolor{colFunc}{rgb}{1,0.07,0.54}
\definecolor{CadetBlue}{cmyk}{0.62,0.57,0.23,0}
\definecolor{Brown}{cmyk}{0,0.81,1,0.60}
\definecolor{colID}{rgb}{0.63,0.44,0}
\newcommand{\passthrough}[1]{#1}
\lstset{defaultdialect=[5.3]Lua}
\lstset{defaultdialect=[x86masm]Assembler}
\lstset{
  basicstyle={\ttfamily\small},
  keywordstyle={\color{OliveGreen}},
  keywordstyle={[2]\color{colFunc}},
  keywordstyle={[3]\color{CadetBlue}},
  commentstyle={\color{Brown}},
  identifierstyle={\color{colID}},
  stringstyle=\color{blue},
  tabsize=2,
  frame=trBL,
  numbers=left,
  numberstyle={\ttfamily\small},
  breaklines=true,
  backgroundcolor={\color[gray]{.95}},
  captionpos=t
}
$endif$
$if(lhs)$
\lstnewenvironment{code}{\lstset{language=Haskell,basicstyle=\small\ttfamily}}{}
$endif$
$if(highlighting-macros)$
$highlighting-macros$
$endif$
$if(verbatim-in-note)$
\usepackage{fancyvrb}
\VerbatimFootnotes % allows verbatim text in footnotes
$endif$
$if(tables)$
\usepackage{longtable,booktabs}
$endif$
$if(graphics)$
\usepackage{graphicx,grffile}
\makeatletter
\def\maxwidth{\ifdim\Gin@nat@width>\linewidth\linewidth\else\Gin@nat@width\fi}
\def\maxheight{\ifdim\Gin@nat@height>\textheight\textheight\else\Gin@nat@height\fi}
\makeatother
% Scale images if necessary, so that they will not overflow the page
% margins by default, and it is still possible to overwrite the defaults
% using explicit options in \includegraphics[width, height, ...]{}
\setkeys{Gin}{width=\maxwidth,height=\maxheight,keepaspectratio}
$endif$
$if(links-as-notes)$
% Make links footnotes instead of hotlinks:
\renewcommand{\href}[2]{#2\footnote{\url{#1}}}
$endif$
$if(strikeout)$
\usepackage[normalem]{ulem}
% avoid problems with \sout in headers with hyperref:
\pdfstringdefDisableCommands{\renewcommand{\sout}{}}
$endif$
$if(indent)$
$else$
\setlength{\parindent}{0pt}
\setlength{\parskip}{6pt plus 2pt minus 1pt}
$endif$
\setlength{\emergencystretch}{3em}  % prevent overfull lines
\providecommand{\tightlist}{%
  \setlength{\itemsep}{0pt}\setlength{\parskip}{0pt}}
$if(numbersections)$
\setcounter{secnumdepth}{5}
$else$
\setcounter{secnumdepth}{0}
$endif$
$if(dir)$
\ifxetex
  % load bidi as late as possible as it modifies e.g. graphicx
  $if(latex-dir-rtl)$
  \usepackage[RTLdocument]{bidi}
  $else$
  \usepackage{bidi}
  $endif$
\fi
\ifnum 0\ifxetex 1\fi\ifluatex 1\fi=0 % if pdftex
  \TeXXeTstate=1
  \newcommand{\RL}[1]{\beginR #1\endR}
  \newcommand{\LR}[1]{\beginL #1\endL}
  \newenvironment{RTL}{\beginR}{\endR}
  \newenvironment{LTR}{\beginL}{\endL}
\fi
$endif$

% $if(title)$
% \title{$title$$if(subtitle)$\\\vspace{0.5em}{\large $subtitle$}$endif$}
% $endif$
% $if(author)$
% \author{$for(author)$$author$$sep$ \and $endfor$}
% $endif$
% \date{$date$}
$for(header-includes)$
$header-includes$
$endfor$


% Inclusion for custom template
\usepackage{tabularx}
\usepackage{ragged2e}
\usepackage[top=35truemm,bottom=30truemm,left=30truemm,right=30truemm]{geometry}
\usepackage{multirow}
\usepackage{gensymb}
\usepackage{textcomp}
% end of inclusion

$if(subparagraph)$
$else$
% Redefines (sub)paragraphs to behave more like sections
\ifx\paragraph\undefined\else
\let\oldparagraph\paragraph
\renewcommand{\paragraph}[1]{\oldparagraph{#1}\mbox{}}
\fi
\ifx\subparagraph\undefined\else
\let\oldsubparagraph\subparagraph
\renewcommand{\subparagraph}[1]{\oldsubparagraph{#1}\mbox{}}
\fi
$endif$

\begin{document}
% $if(title)$
% \maketitle
% $endif$
% $if(abstract)$
% \begin{abstract}
% $abstract$
% \end{abstract}
% $endif$

$for(include-before)$
$include-before$

$endfor$
$if(toc)$
{
\hypersetup{linkcolor=$if(toccolor)$$toccolor$$else$black$endif$}
\setcounter{tocdepth}{$toc-depth$}
\tableofcontents
}
$endif$
$if(lot)$
\listoftables
$endif$
$if(lof)$
\listoffigures
$endif$



% Custom template starts from here
\newenvironment{boldtabular}{ \arrayrulewidth = 2pt }{}
\newenvironment{narrowtabular}{ \renewcommand{\arraystretch}{0.8} }{}
\newenvironment{titletabular}{ \renewcommand{\arraystretch}{1.1} }{}
\newenvironment{datetabular}{ \renewcommand{\arraystretch}{1.1} }{}
\newenvironment{templaturetabular}{ \renewcommand{\arraystretch}{1.18} }{}
\newenvironment{kotitle}{
\centering
  \fontsize{20pt}{20pt}\selectfont
}{}
\fontsize{11pt}{22pt}\selectfont

\vspace*{-13pt}
\begin{kotitle}
  理工学基礎実験レポート

\end{kotitle}

\vspace{19mm}

\begin{titletabular}
\begin{tabularx}{\textwidth}{|m{23mm}|X|}\hline
\parbox[c][9mm][c]{0pt}{}実験日 & $metadata.table1.date$ \\ \hline
\parbox[c][9mm][c]{0pt}{}実験方式  & $metadata.table1.class_type$ \\ \hline
\parbox[c][14mm][c]{0pt}{}実験題目  & $metadata.table1.theme$ \\ \hline
\end{tabularx}
\end{titletabular}

\vspace{8mm}

% \begin{boldtabular}
% \begin{tabularx}{\textwidth}{|l|l|X|l|X|l|X|}\hline
% \parbox[c][10mm][c]{0pt}{} \multicolumn{2}{l}{学科}  & $metadata.table2.faculty$ & クラス & $metadata.table2.class$ & 学籍番号 & $metadata.table2.your_number$ \\  \hline
% \parbox[c][16mm][c]{0pt}{}\textbf{報告者名}  & \multicolumn{6}{l|}{$metadata.table2.your_name$} \\ \hline
% \end{tabularx}
% \end{boldtabular}

\begin{boldtabular}
  \begin{table}[!h]
    \hspace{1pt}
    \begin{tabular}{l}
      \begin{tabularx}{\textwidth}{|l|m{50mm}|l|m{12mm}|l|X|}\hline
        \parbox[c][10.5mm][c]{0pt}{} 学科  & $metadata.table2.faculty$ & クラス & $metadata.table2.class$ & 学籍番号 & $metadata.table2.your_number$ \\
      \end{tabularx}\\

      \begin{tabularx}{\textwidth}{|m{23mm}|X|}\hline
        \centering\fontsize{12pt}{18pt}\selectfont\textbf{報告者氏名}  & \parbox[c][16.5mm][c]{\textwidth}{{\fontsize{20pt}{30pt}\selectfont\textbf{$metadata.table2.your_name$}}} \\ \hline
      \end{tabularx}
    \end{tabular}
  \end{table}
\end{boldtabular}

\vspace{8mm}

\begin{narrowtabular}
\begin{tabularx}{\textwidth}{|m{22mm}|X|X|}\hline
  \multirow{4}{ 22mm }{共同実験者} & $metadata.table3.collaborator1$ & $metadata.table3.collaborator2$ \\ \cline{2-3}
 & $metadata.table3.collaborator3$ & $metadata.table3.collaborator4$ \\ \cline{2-3}
& $metadata.table3.collaborator5$ & $metadata.table3.collaborator6$ \\ \cline{2-3}
& $metadata.table3.collaborator7$ & $metadata.table3.collaborator8$ \\ \hline
\end{tabularx}
\end{narrowtabular}

\vspace{10mm}

\begin{datetabular}
\begin{tabularx}{\textwidth}{|l|X|}\hline
\parbox[c][8mm][c]{0pt}{}レポート提出日 & $metadata.table4.reporting_date$ \\ \hline
\parbox[c][8mm][c]{0pt}{}再レポート提出日 & $metadata.table4.re_reporting_date$ \\ \hline
\parbox[c][8mm][c]{0pt}{} &  \\ \hline
\end{tabularx}
\end{datetabular}

\vspace{12.5mm}

\begin{templaturetabular}
\begin{tabular}{|m{13mm}|m{58mm}|}\hline
\parbox[c][10mm][c]{0pt}{}\centering 室温 & $metadata.table5.temperature$ $if(metadata.table5.temperature)$\celsius $else$ $endif$ \\ \hline
\parbox[c][10mm][c]{0pt}{}\centering 湿度 & $metadata.table5.humidity$ $if(metadata.table5.humidity)$\% $else$ $endif$ \\ \hline
\parbox[c][10mm][c]{0pt}{}\centering 気圧 & $metadata.table5.pressure$ $if(metadata.table5.pressure)$hPa $else$ $endif$ \\ \hline
\end{tabular}
\end{templaturetabular}

\thispagestyle{empty}
\newpage
\setcounter{page}{1}

\normalsize

% end of custom template
$body$

$if(natbib)$
$if(bibliography)$
$if(biblio-title)$
$if(book-class)$
\renewcommand\bibname{$biblio-title$}
$else$
\renewcommand\refname{$biblio-title$}
$endif$
$endif$
\bibliography{$for(bibliography)$$bibliography$$sep$,$endfor$}

$endif$
$endif$
$if(biblatex)$
\printbibliography$if(biblio-title)$[title=$biblio-title$]$endif$

$endif$
$for(include-after)$
$include-after$

$endfor$
\end{document}
